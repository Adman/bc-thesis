\chapter*{Úvod} % chapter* je necislovana kapitola
\addcontentsline{toc}{chapter}{Úvod} % rucne pridanie do obsahu
\markboth{Úvod}{Úvod} % vyriesenie hlaviciek

Monitorovacie systémy nám pomáhajú zbierať dáta, upozorňovať používateľov na rôzne
anomálie, vizualizovať a robiť prehľad historických dát. Dáta, ktoré takto získame nám
vedia poskytnúť veľmi dôležitú spätnú väzbu o fungovaní systému, ktorý monitorujeme.

Táto práca vznikla na podnet vyučujúcich kurzu Linux pre používateľov, ktorým chýbal
takýto prehľad o cvičeniach. V práci sa zameriavame na vývoj aplikácie, ktorá
monitoruje priebeh cvičení, na ktorých študenti riešia úlohy v Linuxovom termináli.
Aplikácia v reálnom čase upozorňuje vyučujúcich na rôzne udalosti
ako napríklad neaktivitu študentov alebo zaseknutia sa v úlohe.
Po cvičení poskytuje rozhranie na vyhodnotenie týchto aktivít. Dôležitou súčasťou
nášho systému je aj podchytenie rôznych spôsobov riešení, ktorými  sa študentom
podarilo vyriešiť zadania.

Práca sa skladá z niekoľkých častí. V kapitole~\ref{kap:specifikacia} uvedieme základné
informácie o monitorovacích systémoch, priblížime fungovanie kurzu Linux pre používateľov
a predstavíme požiadavky na našu aplikáciu.

Kapitola~\ref{kap:navrh} detailne popisuje návrh databázových tabuliek, tok dát a
formát dát, ktoré zbierame od študentov. Taktiež zahŕňa informácie o programovacom jazyku,
knižniciach a frameworkoch, ktoré sme použili pri vývoji.

V kapitole~\ref{kap:implementacia} sa venujeme implementačným detailom aplikácie. V tejto
kapitole taktiež vysvetlíme fungovanie metódy strojového učenia \textit{K-means}, ktorú
využívame v práci.

Kapitola~\ref{kap:testovanie} prezentuje výsledky z testovania aplikácie počas semestra.
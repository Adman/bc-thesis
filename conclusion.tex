\chapter*{Záver}  % chapter* je necislovana kapitola
\addcontentsline{toc}{chapter}{Záver} % rucne pridanie do obsahu
\markboth{Záver}{Záver} % vyriesenie hlaviciek

V tejto práci sa nám podarilo úspešne implementovať systém, ktorý monitoruje priebeh
cvičenia. Systém zbiera dáta od študentov riešiacich úlohy. Tieto dáta sú následne
spracovávané 

Táto aplikácia prináša mnoho nápadov na prácu v budúcnosti. Počas semestra
sme na každom cvičení zozbierali značné množstvo dát, ktoré by sa v budúcich rokoch
dali využiť.
Vzhľadom na tieto historické dáta by bolo dobré naučiť sa, koľko by daná úloha mala
približne trvať (tak ako v počte pokusov tak aj časovo) a na základe toho spraviť
heuristiku na detekciu zaseknutých študentov.
Ďalej by nám tieto historické dáta mohli pomôcť zaradiť študentov do istých
\glqq kategórii problémovosti\grqq~vzhľadom na prílišnú podobnosť riešení s ostatnými,
či príliš veľkú dobu trvania riešenia cvičenia voči ostatným. Týmto študentom by sa
potom mohli vyučujúci o to viac venovať, kedže sa historicky vie, že ich to dokáže
posunúť dopredu.Ďalším zaujímavým vylepšením by bolo implementovať distribúciu cvičiacich
k jednotlivým
študentom. Každý cvičiaci by mal mobilné zariadenie, kam by systém posielal
notifikácie. Systém by si taktiež vedel pamätať, ktorí cvičiaci už úspešne pomohli
študentom a prednostne tých k nim posielal pri rovnakých úlohách.



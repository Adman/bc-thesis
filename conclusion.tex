\chapter*{Záver}  % chapter* je necislovana kapitola
\addcontentsline{toc}{chapter}{Záver} % rucne pridanie do obsahu
\markboth{Záver}{Záver} % vyriesenie hlaviciek

V tejto práci sa nám podarilo úspešne navrhnúť a implementovať systém, ktorý monitoruje
priebeh cvičenia. Systém zbiera dáta od študentov riešiacich úlohy v termináli.
Tieto dáta sú následne spracovávané a v reálnom čase zobrazované vyučujúcim.
Popri zbieraní týchto dát systém automaticky deteguje kritické situácie ako napríklad
zaseknutie študentov na niektorej z úloh alebo príliš veľkú neaktivitu. Vyučujúci
sa tak môžu viac venovať týmto študentom. Taktiež sa nám podarilo naprogramovať
rozhranie, v ktorom je možné hodnotiť riešenia študentov buď ručne alebo automaticky.
Veľkým prínosom je aj možnosť zhlukovania podobných riešení pomocou metódy
\textit{K-means}, ktorá nám poskytuje prehľad o rozmýšľaní študentov.

Systém sa podarilo priebežne testovať na desiatich cvičeniach.
Počas nich sme získavali spätnú
väzbu od vyučujúcich, čo nám pomohlo pri vylepšovaní. Taktiež sa podarilo odhaliť
chyby a pripraviť aplikáciu na plnohodnotné použitie v budúcnosti.

Táto aplikácia prináša mnoho nápadov na prácu v budúcnosti. Počas semestra
sme na každom cvičení zozbierali značné množstvo dát, ktoré by sa v budúcich rokoch
dali využiť.
Vzhľadom na tieto historické dáta by bolo dobré naučiť sa, koľko by daná úloha mala
približne trvať (tak ako v počte pokusov tak aj časovo) a na základe toho spraviť lepšiu
heuristiku na detekciu zaseknutých študentov.
Ďalej by nám tieto historické dáta mohli pomôcť zaradiť študentov do istých
\glqq kategórii problémovosti\grqq~vzhľadom na prílišnú podobnosť riešení s ostatnými,
či príliš veľkú dobu trvania riešenia cvičenia voči ostatným. Týmto študentom by sa
potom mohli vyučujúci o to viac venovať, kedže sa historicky vie, že ich to dokáže
posunúť dopredu. Ďalším zaujímavým vylepšením by bolo implementovať distribúciu cvičiacich
k jednotlivým
študentom. Každý cvičiaci by mal mobilné zariadenie, kam by systém posielal
notifikácie. Aplikáciu by sme chceli rozšíriť aj na iné dómeny výuky
(programovanie v Pythone, C++, Scratchi...), čo by vyžadovalo pridanie odosielania
dát z testovačov a jemnú modifikáciu nášho systému.





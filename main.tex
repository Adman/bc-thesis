\documentclass[12pt, oneside]{book}
\usepackage[a4paper,top=2.5cm,bottom=2.5cm,left=3.5cm,right=2cm]{geometry}
\usepackage[utf8]{inputenc}
\usepackage[T1]{fontenc}
\usepackage{graphicx}
\usepackage{subfig}
\usepackage{url}
\usepackage{enumitem}
\usepackage{makecell}
\usepackage{amsthm}
\usepackage{mathtools}
\usepackage{array}
\usepackage{rotating}
\usepackage{listings,xcolor}
\usepackage[hidelinks,breaklinks]{hyperref}
\usepackage[slovak]{babel} % vypnite pre prace v anglictine
\linespread{1.25} % hodnota 1.25 by mala zodpovedat 1.5 riadkovaniu

% -------------------
% --- Definicia zakladnych pojmov
% --- Vyplnte podla vasho zadania
% -------------------
\def\mfrok{2018}
\def\mfnazov{Inteligentný monitorovací systém práce študentov v Linuxovom termináli}
\def\mftyp{Bakalárska práca}
\def\mfautor{Adrián Matejov}
\def\mfskolitel{RNDr. Marek Nagy, PhD.}

%ak mate konzultanta, odkomentujte aj jeho meno na titulnom liste
\def\mfkonzultant{tit. Meno Priezvisko, tit. }  

\def\mfmiesto{Bratislava, \mfrok}

%aj cislo odboru je povinne a je podla studijneho odboru autora prace
\def\mfodbor{Informatika} 
\def\program{Informatika}
\def\mfpracovisko{Katedra aplikovanej informatiky}

\colorlet{punct}{red!60!black}
\definecolor{delim}{RGB}{20,105,176}
\colorlet{numb}{magenta!60!black}

\lstdefinelanguage{json}{
	basicstyle=\normalfont\ttfamily,
	numbers=left,
	numberstyle=\scriptsize,
	stepnumber=1,
	numbersep=8pt,
	showstringspaces=false,
	breaklines=true,
	frame=lines,
	literate=
	*{0}{{{\color{numb}0}}}{1}
	{1}{{{\color{numb}1}}}{1}
	{2}{{{\color{numb}2}}}{1}
	{3}{{{\color{numb}3}}}{1}
	{4}{{{\color{numb}4}}}{1}
	{5}{{{\color{numb}5}}}{1}
	{6}{{{\color{numb}6}}}{1}
	{7}{{{\color{numb}7}}}{1}
	{8}{{{\color{numb}8}}}{1}
	{9}{{{\color{numb}9}}}{1}
	{:}{{{\color{punct}{:}}}}{1}
	{,}{{{\color{punct}{,}}}}{1}
	{"}{{{\color{delim}{"}}}}{1}
	{\{}{{{\color{delim}{\{}}}}{1}
	{\}}{{{\color{delim}{\}}}}}{1}
	{[}{{{\color{delim}{[}}}}{1}
	{]}{{{\color{delim}{]}}}}{1},
}

\lstdefinelanguage{JavaScript}{
	keywords={typeof, new, true, false, catch, function, return, null, catch, switch, var, if, in, while, do, else, case, break, for},
	keywordstyle=\color{blue}\bfseries,
	ndkeywords={class, export, boolean, throw, implements, import, this},
	ndkeywordstyle=\color{darkgray}\bfseries,
	identifierstyle=\color{black},
	sensitive=false,
	comment=[l]{//},
	morecomment=[s]{/*}{*/},
	commentstyle=\color{purple}\ttfamily,
	stringstyle=\color{red}\ttfamily,
	morestring=[b]',
	morestring=[b]",
	aboveskip=12pt,
}

\lstset{
	language=JavaScript,
	extendedchars=true,
	basicstyle=\small\ttfamily,
	showstringspaces=false,
	showspaces=false,
	frame=single,
	tabsize=2,
	breaklines=true,
	showtabs=true,
	captionpos=b
}

\theoremstyle{definition}
\newtheorem*{prikl}{Príklad}

\newcolumntype{C}[1]{>{\centering\let\newline\\\arraybackslash\hspace{0pt}}m{#1}}
\renewcommand{\thesubfigure}{\arabic{subfigure}}

\begin{document}     
\frontmatter


% -------------------
% --- Obalka ------
% -------------------
\thispagestyle{empty}

\begin{center}
\sc\large
Univerzita Komenského v Bratislave\\
Fakulta matematiky, fyziky a informatiky

\vfill

{\LARGE\mfnazov}\\
\mftyp
\end{center}

\vfill

{\sc\large 
\noindent \mfrok\\
\mfautor
}

\eject % EOP i
% --- koniec obalky ----

% -------------------
% --- Titulný list
% -------------------

\thispagestyle{empty}
\noindent

\begin{center}
\sc  
\large
Univerzita Komenského v Bratislave\\
Fakulta matematiky, fyziky a informatiky

\vfill

{\LARGE\mfnazov}\\
\mftyp
\end{center}

\vfill

\noindent
\begin{tabular}{ll}
Študijný program: & \program \\
Študijný odbor: & \mfodbor \\
Školiace pracovisko: & \mfpracovisko \\
Školiteľ: & \mfskolitel \\
% Konzultant: & \mfkonzultant \\
\end{tabular}

\vfill


\noindent \mfmiesto\\
\mfautor

\eject % EOP i


% --- Koniec titulnej strany


% -------------------
% --- Zadanie z AIS
% -------------------
% v tlačenej verzii s podpismi zainteresovaných osôb.
% v elektronickej verzii sa zverejňuje zadanie bez podpisov

\newpage 
\thispagestyle{empty}
\hspace{-2cm}\includegraphics[width=1.1\textwidth]{images/zadanie}

% --- Koniec zadania

\frontmatter

% -------------------
%   Poďakovanie - nepovinné
% -------------------
\setcounter{page}{3}
\newpage 
~

\vfill
{\bf Poďakovanie:} Chcel by som sa poďakovať môjmu školiteľovi
RNDr. Marekovi Nagyovi, PhD. za poskytnuté konzultácie, cenné rady a možnosť
testovať systém na cvičeniach. Vďaka patrí Marekovi Šuppovi za mnoho nápadov, ochotu
diskutovať o vylepšeniach systému a pomoc s odosielaním dát.
Ďakujem aj mojej rodine a kamarátom, ktorí ma podporovali počas štúdia.

% --- Koniec poďakovania

% -------------------
%   Abstrakt - Slovensky
% -------------------
\newpage 
\section*{Abstrakt}


V tejto práci predstavujeme administratívny a monitorovací systém, ktorý slúži na
monitorovanie priebehu cvičenia, kde študenti pracujú v Linuxovom termináli. Systém sa
snaží automaticky vyhodnocovať kritické situácie a upozorňovať vyučujúcich na neaktivitu
študentov. Takto dostáva učiteľ oveľa lepší prehľad o
napredovaní študentov. Po cvičení ponúka rozhranie na automatické ohodnotenie riešení a
možnosť detekcie rôznych spôsobov riešenia úloh pomocou metódy K-means.

\paragraph*{Kľúčové slová:} monitorovanie cvičenia, Node.js, K-means
% --- Koniec Abstrakt - Slovensky


% -------------------
% --- Abstrakt - Anglicky 
% -------------------
\newpage 
\section*{Abstract}



\paragraph*{Keywords:} 

% --- Koniec Abstrakt - Anglicky


% -------------------
% --- Obsah
% -------------------

\newpage 

\tableofcontents

% ---  Koniec Obsahu

% -------------------
% --- Zoznamy tabuliek, obrázkov - nepovinne
% -------------------

\newpage 

\listoffigures
\listoftables

% ---  Koniec Zoznamov

\mainmatter


\input introduction.tex 

\input specifikacia.tex

\input navrh.tex

\input implementacia.tex

\input testovanie.tex

\input conclusion.tex

% -------------------
% --- Bibliografia
% -------------------


\newpage	

\backmatter

\thispagestyle{empty}
\nocite{*}
\clearpage

\bibliographystyle{plain}
\bibliography{literatura} 

%---koniec Referencii

% -------------------
%--- Prilohy---
% -------------------

%Nepovinná časť prílohy obsahuje materiály, ktoré neboli zaradené priamo  do textu. Každá príloha sa začína na novej strane.
%Zoznam príloh je súčasťou obsahu.
%
\input AppendixA.tex
\addcontentsline{toc}{chapter}{Appendix A}
%
%\addcontentsline{toc}{chapter}{Appendix B}
%\input AppendixB.tex

\end{document}

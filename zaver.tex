\chapter*{Záver}  % chapter* je necislovana kapitola
\addcontentsline{toc}{chapter}{Záver} % rucne pridanie do obsahu
\markboth{Záver}{Záver} % vyriesenie hlaviciek

Na záver už len odporúčania k samotnej kapitole Záver v bakalárskej
práci podľa smernice \cite{smernica}:  \glqq{}V závere je potrebné v
stručnosti zhrnúť dosiahnuté výsledky vo vzťahu k stanoveným
cieľom. Rozsah záveru je minimálne dve strany. Záver ako kapitola sa
nečísluje.\grqq{}

Všimnite si správne písanie slovenských úvodzoviek okolo
predchádzajúceho citátu, ktoré sme dosiahli príkazmi \verb'\glqq' a
\verb'\grqq'.

V informatických prácach niekedy býva záver kratší ako dve strany, ale
stále by to mal byť rozumne dlhý text, v rozsahu aspoň jednej strany.
Okrem dosiahnutých cieľov sa zvyknú rozoberať aj otvorené problémy a
námety na ďalšiu prácu v oblasti.

Abstrakt, úvod a záver práce obsahujú podobné informácie. Abstrakt je
kratší text, ktorý má pomôcť čitateľovi sa rozhodnúť, či vôbec prácu
chce čítať. Úvod má umožniť zorientovať sa v práci skôr než ju začne
čítať a záver sumarizuje najdôležitejšie veci po tom, ako prácu
prečítal, môže sa teda viac zamerať na detaily a využívať pojmy
zavedené v práci.



Táto aplikácia prináša mnoho nápadov na prácu v budúcnosti. Počas semestra
sme na každom cvičení zozbierali značné množstvo dát, ktoré by sa v budúcich rokoch
dali využiť.

Vzhľadom na tieto historické dáta by bolo dobré naučiť sa, koľko by daná úloha mala
približne trvať (tak ako v počte pokusov tak aj časovo) a na základe toho spraviť
heuristiku na detekciu zaseknutých študentov.
Ďalej by nám tieto historické dáta mohli pomôcť zaradiť študentov do istých
\glqq kategórii problémovosti\grqq~vzhľadom na prílišnú podobnosť riešení s ostatnými,
či príliš veľkú dobu trvania riešenia cvičenia voči ostatným. Týmto študentom by sa
potom mohli vyučujúci o to viac venovať, kedže sa historicky vie, že ich to dokáže
posunúť dopredu.

Ďalším zaujímavým vylepšením by bolo implementovať distribúciu cvičiacich k jednotlivým
študentom. Každý cvičiaci by mal mobilné zariadenie, kam by systém posielal
notifikácie. Systém by si taktiež vedel pamätať, ktorí cvičiaci už úspešne pomohli
študentom a prednostne tých k nim posielal pri rovnakých úlohách.



\chapter{Špecifikácia}
\label{kap:specifikacia}

V tejto kapitole spíšeme informácie o textových advetúrach a špecifikáciu
našej aplikácie.

\section{Textové adventúry}
\label{sec:textadventures}

Textové adventúry sú jednými z najstarších typov počítačových hier, ktoré spadajú
pod žáner dobrodružné. Používateľ využíva na interakciu v hre spravidla iba text,
ktorý je vstupom, a hra následne odpovedá taktiež textovým výstupom.
Vstup je väčšinou zadávaný používateľom vo forme krátkych slovných spojení
typu \glqq vziať kľúč \grqq alebo \glqq choď vpred \grqq, ktoré sú spracovávané
textovým parserom.

Prvé parsery rozumeli iba dvojslovným spojeniam typu sloveso + podstatné meno. Neskôr
boli vyvinuté také, ktoré už rozumeli celým vetám.~\cite{bib:adventures}

\subsection{Kurz Linux pre používateľov}
\label{sec:textadventures:linuxforusers}

Kurz Linux pre používateľov sa snaží byť úvodom do operačného systému Linux
na používateľskej úrovni. Študenti sa učia pracovať v termináli so základnými
príkazmi. Vytvárajú priečink, vyhľadávajú a prepisujú určité úseky v texte,
počítajú rôzne štatistické údaje z dát a podobne.

Do akademického roku 2015/2016 sa študenti na cvičeniach snažili vytvárať príkazy v
termináli, a keď si boli istý, že funguje, tak ho skopírovali a poslali na obodovanie.
Neskôr, od roku 2016/2017 vznikol projekt go-term-adventures, pomocou ktorého sa
cvičenia stali viac interkatívnejšími a zefektívnili prácu na nich. V sekcii
~\ref{sec:textadventures:gta} si povieme o tomto projekte viac.

\subsection{go-term-adventures}
\label{sec:textadventures:gta}

Textovými adventúrami sú aj takzvané go-term-adventures (ďalej GTA) (\url{https://github.com/NaiveNeuron/go-term-adventures}), ktoré vznikli
na Fakulte matematiky, fyziky a informatiky UK ako projekt pre kurz Linux pre
používateľov. Hlavným motívom pre tvorbu GTA bolo zefektívnenie cvičení.

Študentov cvičením sprevádza vopred vygenerovaný samorozbaľovací skript. Cvičenie
je rozdelené do takzvaných levelov. Každý level začína vypísaním zadania priamo do
terminálu. Následne je používateľ vyzvaný napísať príkaz v termináli. Ak príkaz
vráti správny výstup, skript pošle používateľa do ďalšieho levelu. Týmto štýlom
študent nemusí prepínať medzi zadaním a terminálom, a taktiež má priamo overenú
správnosť výstupu príkazu.

Skript počas práce študenta zachytáva všetky príkazy a zapisuje ich do textového
dokumentu spoločne s časom vykonania. Ten po cvičení odošlú a prednášajúci ich vie
obodovať.

\section{Motivácia}
\label{sec:motivacia}

V akademickom roku 2016/2017 si kurz Linux pre používateľov na FMFI UK zapísalo
90 študentov. Na cvičeniach sa väčšinou snažilo pomáhať 6 cvičiacich. Študenti riešili
úlohy a často sa stávalo, že potrebovali pomôcť a dlhšiu dobu sa neprihlásili.
Pri tak veľkom počte študentov je ťažké zisťovať, kto sa zasekol na danej úlohe.
Taktiež sa vyskytlo veľa problémov s príliš podobnými riešeniami a nemožnosť ich 
riešiť priamo na cvičení komplikovalo hodnotenie.

Riešenia úloh študenti posielali ako textový súbor so zoznamom príkazov do systému
Moodle a hodnotiaci si ich vždy musel sťahovať aby videl obsah. Človek je
tvor zábudlivý, čo sa potvrdilo aj na tomto kurze. Mnohokrát sa stalo, že študent
zabudol odovzdať tento textový súbor, a tým pádom mu hodnotiaci nemohol prideliť
body za dané cvičenie, čo prinieslo kopu problémov nielen študentovi, ale aj
prednášajúcemu.

Pre naše potreby neexistuje žiadny monitorovací systém, ktorý by spracovával
dáta nášho typu a zároveň ich vizualizoval v reálnom čase podľa potrieb. 
Preto sme sa rozhodli naprogramovať aplikáciu, ktorá bude
monitorovať priebeh cvičenia a cvičiacim tak poskytne prehľad o riešeniach študentov.

\section{Požiadavky na aplikáciu}
\label{sec:apprequirements}

V nasledujúcich bodoch zhrnieme požiadavky na našu aplikáciu

\begin{itemize}
	\item Zbieranie dát z GTA aplikácie od každého jedného študenta
	\item Ukladanie dát do databázy
	\item Zobrazovanie priebehu cvičenia všetkým klientom v reálnom čase
	\item Prehľad historických cvičení
	\item Možnosť obodovať jednotlivé riešenia a exportovať body do csv súboru
	\item Detekcia neaktivity študentov
	\item Detekcia príliš podobných riešení študentov
\end{itemize}

Nižšie sa pozrieme naše požiadavky podrobnejšie.

\subsection{Zbieranie dát z GTA aplikácie od každého jedného študenta}
\label{sec:apprequirements:gtadata}

Aby sme mohli monitorovať aktuálny stav študentov, potrebujeme mať záznam o
všetkých akciách, ktoré študenti vykonali v termináli počas cvičenia.
Dáta budú posielané požiadavkami typu POST v široko používanom JSON formáte.

Každá z požiadaviek bude obsahovať základné informácie o počítači, na ktorom
študent pracuje spolu s dátumom a časom odoslania, typom požiadavky a číslom cvičenia.
Informáciu o IP adrese chceme využívať na to, aby študenti boli nútení
riešiť cvičenie v škole.
Študenti sa prihlasujú na počítače pod svojim univerzitným kontom, ktoré je unikátne.
Tým pádom vieme jednoznačne priradiť študenta k počítaču, čo nám pomôže pri
vizualizácii miestnosti.

\begin{lstlisting}
{
    "type": "typ poziadavky",
    "user": "$USER",
    "hostname": "`hostname`",
    "ip": "lokalna ip pocitaca",
    "exercise_number": cislo cvicenia,
    "date": "datum a cas"
}
\end{lstlisting}

Špecifikujme ďalej typy POST požiadaviek:

\subsubsection{Typ start}
\label{sec:apprequirements:gtadata:start}

Keď študent spustí GTA skript.
Systém si poznačí študenta, ktorý začína riešiť cvičenie.

\begin{lstlisting}
{
    "type": "start",
    "user": "$USER",
    "hostname": "`hostname`",
    "ip": "lokalna ip pocitaca",
    "exercise_number": cislo cvicenia,
    "date": "datum a cas"
}
\end{lstlisting}

\subsubsection{Typ command}
\label{sec:apprequirements:gtadata:command}

Keď študent napíše v temináli príkaz a stlačí enter.
Položka \textit{level} označuje level, v ktorom sa študent nachádza. Položka
\textit{command} označuje odoslaný príkaz.

\begin{lstlisting}
{
    "type": "command",
    "user": "$USER",
    "hostname": "`hostname`",
    "ip": "local computer ip",
    "exercise_number": cislo cvicenia,
    "date": "datum a cas",
    "level": cislo levelu,
    "command": "prikaz z terminalu",
}
\end{lstlisting}

\subsubsection{Typ passed}
\label{sec:apprequirements:gtadata:passed}

Keď študent prejde level. Takisto chceme mať záznam o tom, akým príkazom sa mu podarilo
prejsť daný level.

\begin{lstlisting}
{
    "type": "passed",
    "user": "$USER",
    "hostname": "`hostname`",
    "ip": "local computer ip",
    "exercise_number": cislo cvicenia,
    "date": "datum a cas",
    "level": cislo levelu,
    "command": "prikaz z terminalu",
}
\end{lstlisting}

\subsubsection{Typ exit}
\label{sec:apprequirements:gtadata:exit}

Keď študent ukončí cvičenie napísaním \textit{exit} v termináli.
Systém si poznačí, že daný študent skončil a môže zaradiť jeho riešenie
na ohodnotenie.

\begin{lstlisting}
{
    "type": "exit",
    "user": "$USER",
    "hostname": "`hostname`",
    "ip": "lokalna ip pocitaca",
    "exercise_number": cislo cvicenia,
    "date": "datum a cas"
}
\end{lstlisting}

\subsection{Zobrazovanie priebehu cvičenia všetkým klientom v reálnom čase}
\label{sec:apprequirements:visualization}

Cvičiacim chceme priniesť prehľad o aktuálnom stave všetkých študentov riešiacich
cvičenie. Preto je potrebná pekná vizualizácia v reálnom čase.
Nechceme, aby si museli inštalovať nejaký softvér, preto bude najlepšie využiť webový prehliadač, ktorý je v dnešnej dobe dostupný aj na tabletoch a smartfónoch.

Zároveň pre lepšiu orientáciu cvičiacich v miestnosti ponúkneme možnosť
preusporiadania počítačov v tejto vizualizácii, ktorú si systém zapamätá a
na ďalšom cvičení bude schopný zrekonštruovať toto rozostavenie počítačov v učebni.

\subsection{Prehľad historických cvičení}
\label{sec:apprequirements:overview}

Študenti po skončení kurzu hodnotili tento kurz v študentskej ankete. Veľa z nich
sa vyjadrilo, že niektoré úlohy boli náročné a namiesto jednej vyučovacej hodiny
strávili na cvičení dve až tri.
Je dôležité, aby si tvorcovia jednotlivých úloh na cvičenie vedeli spätne pozrieť
ich náročnosť a poprípade v budúcnosti upraviť zadanie.

\subsection{Možnosť obodovať jednotlivé riešenia a exportovať body do csv súboru}
\label{sec:apprequirements:export}

Kurz prebieha v e-learningovom systéme Moodle. Každý študent si tam vie prezerať
svoje body za cvičenia, testy a podobne. Systém Moodle akceptuje aj takzvaný
csv súbor hodnotenia, ktorý prednášajúci nahrá a body budú automaticky pridelené
študentom. Preto chceme, aby náš systém vedel poskytnúť rozhranie pre obodovanie
jednotlivých úloh, a ich následný export v csv formáte.

\subsection{Detekcia neaktivity študentov}
\label{sec:apprequirements:noactivity}

Študenti sa niekedy zaseknú na úlohe a nevedia ako riešiť daný problém, poprípade
majú problém pochopiť zadanie. Náš systém bude detekovať neaktívnych študentov, ktorí
nevykonali žiadnu akciu po dobu $n$ minút. Systém následne upozorní cvičiacich o tejto
skutočnosti a tí môžu pomôcť študentom

\subsection{Detekcia príliš podobných riešení študentov}
\label{sec:apprequirements:similarsolutions}

Viackrát sa po skončení cvičenia pri hodnotení úloh zistilo, že niektorí študenti
riešili úlohu skupinovo, iní skopírovali celé riešenia od kolegov, iný zas iba časť
riešenia. Takisto ako v predchádzajúcom bode bude systém porovnávať riešenie úlohy
študenta s ostatnými a pokiaľ bude podobnosť príliš veľká, tak bude upozorňovať
cvičiacich na túto skutočnosť.
\chapter{Špecifikácia}
\label{kap:specifikacia}

Táto kapitola obsahuje základné informácie o monitorovacích systémoch a textových
adventúrach, kde všade sme sa s nimi mohli stretnúť, predstavuje základnú motiváciu pre
našu bakalársku prácu a popisuje špecifikáciu našej aplikácie.

\section{Monitorovacie systémy}
\label{sec:monitorsystems}

Monitorovacie systémy sú neoddeliteľnou súčasťou pri vývoji a spravovaní rôznych
komponentov hardvéru, softvéru, mechanických častí a podobne. Dávajú nám dôležité
informácie o ich stave, vedia nás včas upozorňovať o potenciálnych hrozbách,
zbierajú dáta, ktoré vieme vyhodnocovať a predísť podobným prípadom v budúcnosti.
Priblížme si niektoré typy monitorovacích systémov.

\textit{Sieťové} sledujú dostupnosť webových služieb, mailových serverov, využitie
diskového priestoru, oneskorenie odpovede alebo možný DDoS útok. Následne vedia informovať administrátora o týchto skutočnostiach. Príkladom takéhoto systému je \textit{Zabbix}.
~\cite{bib:networkmonitoring}

\textit{Softvérové} zbierajú informácie o vyťažovaní procesorov, RAM pamäte, celkovom
stave aplikácií. V unixových systémoch je na vizualizáciu týchto informácií využívaný
program \textit{top}, respektíve \textit{htop}.

\textit{Hardvérové} sú využívané napríklad na sledovanie teploty zariadenia, funkčnosti rôznych mechanických častí, tlaku v nádržiach s kvapalinami atď. 

Na monitorovanie \textit{počasia} sú taktiež využívané rôznorodé senzory a satelity.
Pomocou nich vieme získavať cenné dáta o aktuálnom stave počasia takmer hocikde
na zemi, ktoré sú zužitkované na predpovede alebo varovanie obyvateľstva pred
blížiacou sa prírodnou katastrofou.

Všetky tieto systémy majú dôležité spoločné črty. Vedia zbierať dáta, spracovávať ich,
vyhodnocovať, vizualizovať a informovať administrátorov a používateľov o neštandardnom
správaní. My sa v tejto práci budeme venovať taktiež monitorovaciemu systému, ktorý sleduje aktivitu študentov na cvičení.

\section{Textové adventúry}
\label{sec:textadventures}

Textové adventúry sú jednými z najstarších typov počítačových hier, ktoré spadajú
pod dobrodružný žáner. Používateľ využíva na interakciu v hre spravidla iba klávesnicu
a hra následne odpovedá textovým výstupom.
Vstup je väčšinou zadávaný používateľom vo forme krátkych slovných spojení
typu \glqq vziať kľúč\grqq~alebo \glqq choď vpred\grqq, ktoré sú spracovávané
textovým parserom. Prvé parsery rozumeli iba dvojslovným spojeniam typu sloveso + podstatné meno. Neskôr boli vyvinuté také, ktoré už rozumeli celým vetám.

Napriek neprítomnosti grafického prostredia môžu tieto typy hier popisovať prostredie, v ktorom sa hráč pohybuje, čo rozvíja predstavivosť. Taktiež môže vytvárať nelogické
prostredie, kde z bodu A sa dá dostať do bodu B, ale z bodu B sa už reverzným pohybom
nedostaneme do bodu A.~\cite{bib:adventures}


\section{Motivácia}
\label{sec:motivacia}

\subsection{go-term-adventures}
\label{sec:motivacia:gta}

Textovými adventúrami sú aj takzvané \textit{go-term-adventures} (ďalej GTA) (\url{https://github.com/NaiveNeuron/go-term-adventures})~\cite{bib:gta}, ktoré vznikli
na Fakulte matematiky, fyziky a informatiky Univerzity Komenského ako projekt pre kurz Linux pre používateľov. Hlavným motívom pre tvorbu GTA bolo zefektívnenie cvičení.
Ako už samotný názov napovedá, GTA aplikácia je naprogramovaná v programovacom
jazyku Go.

Študentov cvičením sprevádza vopred vygenerovaný samorozbaľovací skript. Cvičenie
je rozdelené do takzvaných \textit{levelov}. Každý level začína vypísaním zadania priamo do
terminálu. Následne je používateľ vyzvaný napísať príkaz v termináli. Ak zadaný príkaz
vráti správny výstup, skript pošle používateľa do ďalšieho levelu. Vďaka tomu
študent nemusí prepínať medzi zadaním a terminálom, a taktiež má priamo overenú
správnosť výstupu príkazu. Niektoré levely majú viac alternatív, kde každá
má jemne pozmenené zadanie. Študenti riešia iba jednu alternatívu každého z levelov.

Skript počas práce študenta zachytáva všetky príkazy a zapisuje ich do textového
dokumentu spoločne s časom vykonania. Ten po cvičení odošlú a prednášajúci ich vie
obodovať.

\subsection{Kurz Linux pre používateľov}
\label{sec:motivacia:linuxforusers}

Kurz Linux pre používateľov je úvodom do operačného systému Linux
na používateľskej úrovni. Študenti sa učia pracovať v termináli so základnými
príkazmi. Vytvárajú priečinky, vyhľadávajú a prepisujú určité úseky v texte,
počítajú rôzne štatistické údaje z dát a podobne.

Do akademického roku 2015/2016 študenti na cvičeniach vytvárali príkazy v
termináli, a keď si boli istí, že fungujú, tak ich skopírovali a poslali na obodovanie.
Neskôr, od roku 2016/2017 vznikol projekt go-term-adventures
(viď sekciu \ref{sec:motivacia:gta}), vďaka ktorému sa
cvičenia stali viac interaktívnejšími a aj práca na nich sa zefektívnila.

V akademickom roku 2016/2017 si kurz Linux pre používateľov na FMFI UK zapísalo
90 študentov. Na cvičeniach sa väčšinou snažilo pomáhať 6 cvičiacich. Študenti riešili
úlohy a často sa stávalo, že potrebovali pomôcť a dlhšiu dobu sa neprihlásili.
Pri tak veľkom počte študentov je ťažké zisťovať, kto sa zasekol na danej úlohe.
Taktiež sa vyskytlo veľa problémov s príliš podobnými riešeniami a nemožnosť riešiť podozrenie z opisovania priamo na cvičení komplikovalo hodnotenie.

Riešenia úloh študenti posielali ako textový súbor so zoznamom príkazov do systému
Moodle a hodnotiaci si ich vždy musel sťahovať, aby videl obsah. Človek je
tvor zábudlivý, čo sa potvrdilo aj na tomto kurze. Mnohokrát sa stalo, že študent
zabudol odovzdať tento textový súbor, a tým pádom mu hodnotiaci nemohol prideliť
body za dané cvičenie, čo prinieslo mnoho problémov nielen študentovi, ale aj
prednášajúcemu.

Pre naše potreby neexistoval žiadny monitorovací systém, ktorý by spracovával
dáta nášho typu a zároveň ich vizualizoval v reálnom čase. 
Preto sme sa rozhodli naprogramovať aplikáciu, ktorá bude
monitorovať priebeh cvičenia, a vyučujúcim tak poskytne prehľad nielen o riešeniach
študentov ale aj administrátorské webové rozhranie na správu cvičení.

\section{Požiadavky na aplikáciu}
\label{sec:apprequirements}

V nasledujúcich bodoch zhrnieme požiadavky na našu aplikáciu:

\begin{itemize}
	\item Zbieranie dát z GTA aplikácie od všetkých študentov
	\item Ukladanie dát do databázy
	\item Zobrazovanie priebehu cvičenia všetkým klientom (vyučujúcim) v reálnom čase
	\item Prehľad historických cvičení
	\item Automatické obodovanie jednotlivých riešení a exportovanie bodov do csv súboru
	\item Detekcia neaktivity študentov
	\item Detekcia rôznych spôsobov riešenia úloh
\end{itemize}

Nižšie sa pozrieme na naše požiadavky podrobnejšie.

\subsection{Zbieranie dát z GTA aplikácie od všetkých študentov}
\label{sec:apprequirements:gtadata}

Aby sme mohli monitorovať aktuálny stav študentov, potrebujeme mať záznam o
všetkých akciách, ktoré študenti vykonali v termináli počas cvičenia.

Každá z požiadaviek bude obsahovať základné informácie o počítači, na ktorom
študent pracuje spolu s dátumom a časom odoslania, typom požiadavky a číslom cvičenia.
Informáciu o IP adrese chceme využívať na to, aby študenti boli nútení
riešiť cvičenie v škole.
Študenti sa prihlasujú na počítače pod svojim univerzitným kontom, ktoré je unikátne.
Tým pádom vieme jednoznačne priradiť študenta k počítaču, čo nám pomôže pri
vizualizácii miestnosti.


\subsection{Zobrazovanie priebehu cvičenia všetkým klientom (vyučujúcim) v reálnom čase}
\label{sec:apprequirements:visualization}

Cvičiacim chceme priniesť prehľad o aktuálnom stave všetkých študentov riešiacich
cvičenie. Preto je potrebná pekná vizualizácia v reálnom čase.
Nechceme, aby si museli inštalovať nejaký softvér, preto bude najlepšie využiť webový prehliadač, ktorý je v dnešnej dobe dostupný aj na tabletoch a smartfónoch.
Zobrazovať chceme hlavne jednoznačný identifikátor študentov a počítačov, na ktorých
pracujú, level, v ktorom
sa aktuálne nachádzajú, čas od posledného príkazu, neaktivitu po dlhšej dobe,
a taktiež počet pokusov pri prechádzaní aktuálneho levelu. Bude sa dať pozrieť
aj detail o študentovi, kde zobrazíme čas začiatku, konca a všetky príkazy, ktoré
napísal na cvičení.

Zároveň pre lepšiu orientáciu vyučujúcich v miestnosti ponúkneme možnosť
preusporiadania počítačov v tejto vizualizácii, ktorú si systém zapamätá a
na ďalšom cvičení bude schopný zrekonštruovať toto rozostavenie počítačov v učebni.

\subsection{Prehľad historických cvičení}
\label{sec:apprequirements:overview}

Študenti po skončení kurzu hodnotili tento kurz v študentskej ankete. Veľa z nich
sa vyjadrilo, že niektoré úlohy boli náročné a namiesto jednej vyučovacej hodiny
strávili na cvičení dve až tri.
Je dôležité, aby si tvorcovia jednotlivých úloh na cvičenie vedeli spätne pozrieť
ich náročnosť a poprípade v budúcnosti upraviť zadanie.
Zaujímavé dáta, ktoré chceme o každom študentovi zobraziť, sú čas strávený riešením cvičenia a počet príkazov. Ďalej sú to priemery a mediány časov riešenia všetkých študentov a počtu príkazov.


\subsection{Automatické obodovanie jednotlivých riešení a exportovanie bodov do csv súboru}
\label{sec:apprequirements:export}

Kurz prebieha v e-learningovom systéme Moodle. Každý študent si tam vie prezerať
svoje body za cvičenia, testy, nájde tam prednášky a iné materiály. Systém Moodle akceptuje aj formát
\textit{csv}, v ktorom prednášajúci nahrá súbor hodnotenia a body budú automaticky pridelené
študentom. Preto chceme, aby náš systém vedel poskytnúť rozhranie pre automatické
obodovanie jednotlivých úloh a ich následný export v csv formáte.

\subsection{Detekcia neaktivity študentov}
\label{sec:apprequirements:noactivity}

Študenti sa niekedy zaseknú na úlohe a nevedia ako riešiť daný problém, poprípade
majú problém pochopiť zadanie. Náš systém bude detegovať neaktívnych študentov, ktorí
nevykonali žiadnu akciu po dobu $n$ minút. Systém následne upozorní vyučujúcich o tejto
skutočnosti a tí môžu pomôcť študentom.
Taktiež bude upozorňovať vyučujúcich, pokiaľ sa študentovi nepodarí prejsť level
po určitom počte príkazov.

\subsection{Detekcia rôznych spôsobov riešenia úloh}
\label{sec:apprequirements:similarsolutions}

Veľa z úloh, ktoré sa vyskytujú na cvičeniach má viacero spôsobov riešenia. Vo väčšine
prípadov aj študenti nerozmýšľajú úplne rovnako pri riešení. Preto chceme vyučujúcemu
poskytnúť cennú spätnú väzbu o týchto spôsoboch. Získa tak prehľad
o tom, či študenti porozumeli prednáške a v budúcnosti sa tak môže zamerať na zlepšenie
výkladu učiva, poprípade sa viac sústrediť na to, čo študentom nešlo. Taktiež
tento prehľad môže byť použitý pri podozrení na plagiátorstvo.